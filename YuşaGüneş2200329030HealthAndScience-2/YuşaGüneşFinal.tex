\documentclass[10pt,xcolor=dvipsnames]{beamer}

\usetheme[progressbar=frametitle]{metropolis}

\usepackage{booktabs}
\usepackage[scale=2]{ccicons}

\usepackage{pgfplots}
\usepgfplotslibrary{dateplot}

\usepackage{xspace}
\newcommand{\themename}{\textbf{\textsc{metropolis}}\xspace}

\usepackage{pdfpages}
\setbeamercolor{section title}{fg=Maroon,bg=Maroon}
\setbeamercolor*{structure}{bg=Maroon!20,fg=Maroon}
\setbeamercolor*{palette primary}{use=structure,fg=white,bg=structure.fg}
\setbeamercolor{progress bar}{fg=gray, bg=gray}

%%%%%%%%%%%%%%%%%%%%%%%%%%%%%%%%%%%%%%%%%%%%%%%%%%%%%%%%%%%%%%%%%%%%%%

\title{İST480 FİNAL ÖDEVİ}
\subtitle{Hacettepe Üniversitesi}
\date{Haziran 2023}
\author{Yuşa Faik Güneş}
\institute{}

%%%%%%%%%%%%%%%%%%%%%%%%%%%%%%%%%%%%%%%%%%%%%%%%%%%%%%%%%%%%%%%%%%%%%%

\begin{document}

\maketitle

\begin{frame}{Makalenin Yayınlandığı Dergi Hakkında Genel Bilgiler}
\begin{block}{Dergi Adı}
Health and Technology
\end{block}


\begin{block}{Dergi Indeksi}
Tr Dizin , Science Citation Index Expanded (SCI-EXPANDED) , SIndexed by Science Citation Index, Inspec, Compendex, DBLP, Computer Science Index, Current Abstracts, Current Contents, EBSCO host, IngentConnect, MetaPress, Springerlink, OCLC
\end{block}

  
  \begin{block}{Dergi Hakkında}
  Health and Technology, Springer ve IUPESM (International Union for Physical and Engineering Sciences in Medicine) tarafından Dünya Sağlık Örgütü işbirliği ile ortaklaşa yayınlanmaktadır. Sağlık, bakım ve sağlık teknolojisi ile ilgili tüm mesleklere hitap eden sağlık teknolojileri ile ilgili konularda gerçek anlamda disiplinler arası ilk dergidir.

  \end{block}
  \begin{block}{Derginin erişim linki}
https://www.springer.com/journal/12553
 \end{block}
  
\end{frame}

\begin{frame}
\frametitle{Makale İncelemesi}


\begin{block}{Makale Adı}
A survey of deep learning for MRI brain tumor segmentation methods: Trends, challenges, and future directions
\end{block}

\begin{block}{Yazarlar}
Srigiri Krishnapriya , Yepuganti Karuna 
\end{block}

\begin{block}{Makale Dili}
İngilizce
\end{block}

\end{frame}



\begin{frame}{Makalenin konusu}

Bu makalenin konusu beyin tümörü segmentasyonudur. Beyin tümörü segmentasyonu, farklı tümör dokularını (katı veya aktif tümör, ödem ve nekroz) normal beyin dokularından ayırmayı amaçlar. Makale, beyin tümörü çalışmalarında anormal dokuların doğru ve tekrarlanabilir şekilde segmentasyonunu ve karakterizasyonunu ele almaktadır.

\end{frame}


\begin{frame}
\frametitle{Makalenin Amacı}
beyin tümörü segmentasyonu alanında kullanılan yöntemlere genel bir bakış sunmaktır. Hem yarı otomatik hem de tam otomatik yöntemlerin kullanıldığı segmentasyon teknikleri incelenir. Ayrıca, manyetik rezonans görüntülemenin diğer tanısal görüntüleme yöntemlerine göre avantajları göz önüne alınarak, araştırma MRG beyin tümörü segmentasyonuna odaklanmaktadır
\end{frame}


\begin{frame}
\frametitle{Makalenin Önemi }
\textbf{Klinik Uygulanabilirlik:}

Beyin tümörü segmentasyonu, tıbbi görüntüleme alanında büyük öneme sahiptir. Doğru ve tekrarlanabilir bir şekilde beyin tümörlerinin segmente edilmesi, tanı ve tedavi süreçlerinde büyük bir rol oynamaktadır. Bu makale, beyin tümörü segmentasyonu için kullanılan farklı yöntemleri inceleyerek, klinik uygulanabilirliklerini değerlendirir. Bu da tıbbi uzmanlara ve araştırmacılara uygun yöntemleri seçme konusunda rehberlik eder.
\end{frame}

\begin{frame}
\frametitle{Makalenin Önemi }
\textbf{Otomatik Segmentasyon:}

Makale, hem yarı otomatik hem de tam otomatik beyin tümörü segmentasyonu tekniklerini ele almaktadır. Otomatik segmentasyon teknikleri, zaman tasarrufu sağlar ve insan hatasını azaltabilir. Bu, tıbbi görüntüleme çalışmalarında önemli bir avantajdır. Makalede sunulan farklı otomatik teknikler, segmentasyon sürecini daha verimli hale getirmek için incelenir.
\end{frame}



\begin{frame}
\frametitle{Örnekleme Planı}
örnekleme planı, beyin tümörü görüntüleme analizi için kullanılan veri seçim ve işleme yöntemlerini ve ilgili teknikleri içeren bir planı ifade etmektedir.

\end{frame}

\begin{frame}
\frametitle{Kullanılan Yöntemler}
Görüntü İşleme Ve Analizi:Beyin tümörü görüntüleme verileri üzerinde çeşitli görüntü işleme ve analiz yöntemleri kullanılabilir. Bunlar arasında filtreleme, segmentasyon, kenar tespiti, yoğunluk ölçümü, şekil analizi, histogram eşitleme gibi teknikler yer alabilir. Bu yöntemler, tümör bölgelerini belirlemek, boyutlarını ölçmek ve farklı dokuları ayırmak için kullanılabilir.
\end{frame}

\begin{frame}
\frametitle{Kullanılan Yöntemler}
Veri Etiketleme Ve Sınıflandırma:Beyin tümörü görüntüleme verileri üzerinde sınıflandırma yapmak için makine öğrenimi yöntemleri kullanılabilir. Örneğin, destek vektör makineleri (SVM), yapay sinir ağları (YSA), karar ağaçları gibi algoritmalar kullanılarak görüntülerdeki farklı dokular ve tümör bölgeleri sınıflandırılabilir.
\end{frame}

\begin{frame}
\frametitle{Kullanılan Yöntemler}
Öznitelik Çıkarma:Beyin tümörü görüntüleme verilerinden önemli özniteliklerin çıkarılması önemlidir. Bu öznitelikler, tümörün boyutu, şekli, yoğunluğu gibi özellikleri ifade edebilir. Öznitelik çıkarma yöntemleri, tümör bölgelerini tanımlamak ve farklı dokular arasındaki farkları belirlemek için kullanılabilir.
\end{frame}

\begin{frame}
\frametitle{Sonuç}
Beyin tümörü görüntüleme araştırmalarında temel bir hedef, kanseri doğru bir şekilde lokalize etmektir. Tümörler, normal dokulardan ayırt edilmesine olanak tanıyan özelliklere göre segmentasyon teknikleriyle analiz edilmektedir. Bazı tümörler, görüntü yoğunluklarıyla normal dokulardan ayırt edilebildiğinden, eşikleme veya bölge büyütme gibi teknikler kullanılmıştır. Diğer tümörler ise şekilleriyle tanımlanabildiğinden model tabanlı bir yöntem kullanılmıştır.

Önerilen otomatik yöntemlerin beyin tümörü segmentasyonundaki rapor edilen doğruluğu oldukça umut verici olsa da, bu yaklaşımlar hala patologlar arasında günlük klinik uygulama için geniş kabul görmemiştir. Bunun başlıca nedenleri standartlaşmış prosedürlerin eksikliği olabilir. Diğer iki neden, geleneksel uzmanların çalışma şekilleriyle önemli farklılıklar ve mevcut yöntemlerin tıbbi karar desteklemesine şeffaf ve yorumlanabilir bir şekilde yardımcı olmamasıdır. Son iki faktör, nedenlendirme ve açıklamanın ana öncelik olduğu bilgisayar destekli tıbbi teşhis için çok önemlidir.
\end{frame}

\begin{frame}
\frametitle{Makale Hakkında Görüşler}
\begin{block}{Makalenin Artı Yönleri}
 Klinikte yapılan çalışmalara göre daha hızlı bir sonuç çıkartır.
\end{block}
\end{frame}
\begin{frame}
\frametitle{Makale Hakkında Görüşler}
\begin{block}{Makalenin Eksi Yönleri}
Beyin tümörü segmentasyonu klinik çalışmalara göre güvenirliği az ve pataloglara göre geniş bir kapsam sunmaktadir.
\end{block}
\end{frame}
\begin{frame}
\frametitle{References}

\begin{itemize}
    \item[\textbf{[1]}] Wong K. \emph{Medical image segmentation: methods and applications in functional imaging.} \emph{Handb Biomed Image Anal Segmentation Models Part B} 2005;2: 111--82.
    
    \item[\textbf{[2]}] Bhandarkar S, Koh J, Suk M. \emph{Multiscale image segmentation using a hierarchical self-organizing map.} \emph{Neurocomputing} 1997;14:241--72.
    
    \item[\textbf{[3]}] Corso J, Sharon E, Dube S, El-Saden S, Sinha U, Yuille A. \emph{Efficient multilevel brain tumor segmentation with integrated Bayesian model classification.} \emph{IEEE Trans Med Imaging} 2008;27(5):629--40.
    
    \item[\textbf{[4]}] Yao J. \emph{Image processing in tumor imaging.} In: \emph{New techniques in oncologic imaging;} 2006. p. 79--102.
\end{itemize}

\end{frame}
\end{document}
